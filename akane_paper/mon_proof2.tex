%% write up of monitor algorithm
% Aaron Kane

% at eventually, need to finish writing up safety_monitor()
% should also decide on notation for append, step/bound #, etc

\documentclass[10pt,a4paper]{article}
\usepackage[noend]{algorithmic}
\usepackage{algorithm}
\usepackage[latin1]{inputenc}
\usepackage{amsmath}
\usepackage{amsfonts}
\usepackage{amsthm}
\usepackage{amssymb}
\usepackage{stmaryrd}
\usepackage{bussproofs}

%\addtolength{\textheight}{1in}
%{\voffset}{-1.5in}

\newcommand{\smon}[1]{\ensuremath{smon(\sigma, \tau, #1)}}
\newcommand{\ty}[1]{\texttt{#1}}
\newcommand{\bptr}{b}
\newcommand{\fptr}{f}
\newcommand{\ov}[1]{\ensuremath{\overline{#1}}}
\newcommand{\ep}{ep}
\newcommand{\cp}{cp}
\newcommand{\rp}[2]{\ensuremath{\langle #1, #2 \rangle}}
\newcommand{\res}[2]{\ensuremath{r_{#1}^{#2}}}
\newcommand{\agmon}{\ensuremath{\mathbf{agmon}}}
\newcommand{\pst}{\ensuremath{S^i_\psi}}
\newcommand{\rpt}[3]{\ensuremath{\langle #1, #2 \rangle}_{#3}}



\newtheorem{thm}{Theorem}
\newtheorem{tdef}{Definition}
\newtheorem{lemma}{Lemma}
\newtheorem{case}{Case}

\title{Aggressive Monitor Proof Writeup}
\author{Aaron Kane}

\begin{document}
\maketitle

\section{Definitions}

A residue $\rp{j}{\psi}_\phi$ is a tagged pair where $j \in \mathbb{N}$ is a position in the log and $\psi, \phi$ are well-formed formulas.

A history structure $S_{\phi}^i = \{ r_{\phi}^0, r_{\phi}^1, \ldots r_{\phi}^i \}$ is a list of residues.

We let $\mathbb{S}^i_{\psi}$ represent the set of history structures for all temporal subformula of $\psi$, i.e. $\mathbb{S}^i_{\phi} = \bigcup_{\psi \in \mathbf{tempSub}(\phi)} S^i_\psi$. 

\begin{tdef}[Correctness of history structures] For all $i, \psi, \sigma, \tau$, a history structures $S^i_{\psi}$ is correct if $\forall \rpt{j}{\psi'}{\psi} \in \pst.(\sigma, \tau, j \vDash \psi' \leftrightarrow \sigma, \tau, j \vDash \psi)$.
\end{tdef}
\begin{tdef}[Promptness of history structures] For all positions $i$, formulas $\psi$, traces $\sigma$, and time sequences $\tau$, a history structures $S^i_{\psi}$ is prompt if $\forall \rpt{j}{\psi'}{\psi} \in \pst.(\tau_i-\tau_j \geq \Delta(\psi) \rightarrow \psi' \equiv \top/\bot)$
\end{tdef}

%\begin{tdef}[Correctness of history structures] For all positions $i$, formulas $\psi$, traces $\sigma$, and time sequences $\tau$, a history structures $S^i_{\psi}$ is correct if:
%\begin{enumerate}
%	\item \text{Sound: } $\forall \rpt{j}{\psi'}{\psi} \in \pst.(\sigma, \tau, j \vDash \psi' \rightarrow \sigma, \tau, j \vDash \psi)$
%	\item \text{Complete: } $\forall j \leq i.(\sigma, \tau, j \vDash \psi \rightarrow \exists \rp{j}{\psi'} \in \pst. \sigma, \tau, j \vDash \psi'$ \text{ and } $\sigma, \tau, j \nvDash \psi \rightarrow \exists \rp{j}{\psi'} \in \pst. \sigma, \tau, j \nvDash \psi')$
%	\item \text{Prompt: } $\forall \rp{j}{\psi'} \in \pst.(\tau_i - \tau_j \geq \Delta^w(\psi) \rightarrow \psi' \equiv \top/\bot)$
%\end{enumerate}
%\end{tdef}


With these properties defined, we ultimately wish to prove the correctness of our top-level algorithm \agmon.

\begin{thm}[Correctness of \agmon]
For all formula $\varphi$, all $i \in \mathbb{N}$, all time stamp sequences $\tau$, and all traces $\sigma$ it is the case that if $S^i_{\varphi}$ is 1) sound, 2) complete, and 3) prompt then $\exists j.(\rp{j}{\bot} \in S^i_{\varphi}) \leftrightarrow \sigma, \tau, j \nvDash \varphi$
\end{thm}

\begin{thm}[Correctness of \agmon 2]
For all $i \in \mathbb{N}$, all formula $\varphi$, all time stamp sequences $\tau$ and all traces $\sigma$ it is the case that if $S^{i-1}_{\varphi}$ is correct then after the loop, $S^{i}_{\varphi}$ will be correct.
\end{thm}
\textit{Proof.} We must show that given a correct $\mathbb{S}^i_{\psi}$ and $S^{i-1}_{\psi}$ that \agmon returns a correct $S^i_{\psi}$. Since $S^i_{\psi}$ is just the union of the reduction of all residues in $S^{i-1}_{\psi}$ and the additional reduced residue at step $i$ we know that $S^i_{\psi}$ is sound and complete since reduce is correct (Lemma 2) and there will be a residue for all $j \leq i$ included ($S^{i-1}$ provides all $j < i$ and we add $i$). $S^i_{\psi}$ is also prompt since reduce is correct.

\subsection{Proof of \agmon\ Correctness}
In this section we present some assisting lemmas and definitions we will use to show the correctness of \agmon.

\begin{lemma}[Correctness of $\mathbf{simplify}(\psi)$] For all positions $i$, formulas $\psi$, traces $\sigma,$ and time sequences $\tau$ 
it is the case that if $\psi' \leftarrow \mathbf{simplify}(\psi)$ then $\sigma, \tau, i \vDash \psi \leftrightarrow \sigma, \tau, i \vDash \psi'$

Proof. Trivial by the semantics of $\mathbf{simplify}$.
\end{lemma}


\begin{lemma}[Correctness of $\mathbf{reduce}(\sigma, \tau, i, \mathbb{S}^i_{\psi}, \rpt{j}{\psi'}{\psi})$]
For all $i, j \leq i, \psi, \sigma, \tau,$ correct $\mathbb{S}^i_{\psi}$ and $\psi'$ such that $\sigma, \tau, j \vDash \psi \leftrightarrow \sigma, \tau, j \vDash \psi'$ then if $\rp{j'}{\phi} \leftarrow \mathbf{reduce}(\sigma,\tau,i,\mathbb{S}^i_{\psi}, \rpt{j}{\psi'}{\psi})$ then $\sigma, \tau, j \vDash \psi' \leftrightarrow \sigma, \tau, j \vDash \phi$.
\end{lemma}

\noindent Proof by induction over $\psi$. \\

%%%%%% PROP
\noindent \textbf{Case $\psi \equiv p$} \\
There are two cases: $\sigma, \tau, j \vDash p \rightarrow \sigma, \tau, j \vDash \phi$ and $\sigma,\tau,j\vDash \phi \rightarrow \sigma,\tau,j\vDash p$.

\textbf{Subcase 1}
From the semantics of $p$ in this case we know that $\sigma_i(p) = \top$. In this case $\phi = \top$ (since $\sigma_i(p) = \top$ so we have $\sigma,\tau,j\vDash p \rightarrow \sigma,\tau,j\vDash \top$ which is trivially true).

\textbf{Subcase 2}
There are two possible subcases here, either $\sigma_i(p) = \bot$ or $\sigma_i(p) = \top$. If $\sigma_i(p) = \bot$ then trivially $\sigma,\tau,j\vDash \bot \rightarrow \sigma,\tau,j\vDash p$. If $\sigma_i(p) = \top$ then we have $\sigma,\tau,j\vDash \top \rightarrow \sigma,\tau,j\vDash p$ which is true from the semantics of $p$ (since $\sigma_i(p) = \top$).
\\ \\
%%%%% SINCE
\noindent \textbf{Case $\alpha \mathcal{S}_{[l,h]} \beta$}
There are two cases, 1) $\sigma, \tau, j \vDash \psi' \rightarrow \sigma, \tau, j \vDash \phi$ and 2) $\sigma, \tau, j \vDash \phi \rightarrow \sigma, \tau, j \vDash \psi'$.

\textbf{Subcase [$\sigma, \tau, j \vDash \psi' \rightarrow \sigma, \tau, j \vDash \phi$]} 
By the semantics of Since, we know that $\exists k \in [\tau_j-h,\tau_j-l].(\sigma, \tau, k \vDash \beta \wedge \forall k' \in [k,j].(\sigma, \tau, k' \vDash \alpha))$. Since $S^i_{\beta}$ is correct then there is a $\rp{k}{\top} \in S^i_{\beta}$ and since $S^i_\alpha$ is correct then $\forall k' \in [k+1,j]. \rp{k'}{\top} \in S^i_\alpha$.

Given these values in the history structs, we can see from the definition of reduce that $a_a = k$, $a_s = k$, $b_a \geq k$, $b_t \geq k$, and $b_n = \bot$. With these values, we see that $a_s \leq b_t + 1$ and so $\mathbf{reduce}$ returns $\rp{j}{\top}$ and so this case is satisfied.

\textbf{Subcase [$\sigma, \tau, j \vDash \phi \rightarrow \sigma, \tau, j \vDash \psi'$]}
There are three possible subcases here: $\phi = \top$, $\phi = \bot$, and $\phi = \psi'$.

\hspace{2em} \textbf{Sub-Subcase} $\phi = \top$
If $\phi = \top$ then we know that $a_s \leq b_t + 1$. For this to be the case there exists $k$ such that $\tau_k \in [\tau_j-h,\tau_j-l] \wedge \rp{k}{\top} \in S^i_\beta$ (i.e., $b_t = k$) and there exists $a \leq k+1$ such that $\forall a' \in [\tau_{k+1},\tau_j].(\rp{a'}{\top} \in S^i_\alpha)$ (i.e., $a_s = a$). Since $\rp{k}{\top}$ is in $S^i_\beta$ and $S^i_\beta$ is correct, we know that $\sigma,\tau,k \vDash \beta$. Since $S^i_\alpha$ is correct we also know from $a_s$ that $\forall k' \in [\tau_{k+1},\tau_j] \sigma, \tau, k' \vDash \alpha$. Given this, we can see that the semantics for Since are met and so $\sigma, \tau, j \vDash \alpha\, \mathcal{S}_{[l,h]}\, \beta$.

\hspace{2em} \textbf{Sub-Subcase} $\phi = \bot$
If $\phi = \bot$ then we know that either $b_n = \top$ or $a_a > b_a$. If $b_n = \top$ then $\tau_i-\Delta(\psi) \geq \tau_j) \wedge \forall k.(\tau_k \in [\tau_j-l,\tau_j-h] \rightarrow \rp{k}{\bot} \in S^i_\beta)$. Since $S^i_\beta$ is correct, $\forall k.(\tau_k \in [\tau_j-l,\tau_j-h]. \sigma, \tau, k \nvDash \beta)$ which means that $\sigma, \tau, j \nvDash \alpha\, \mathcal{S}_{[l,h]}\, \beta$ due to the semantics of Since.

In the other case, we have $a_a > b_a$. From $b_a$ we know that $\forall k.(k \in [b_a,\tau_j-l] \rightarrow \rp{k}{bot} \in S^i_\beta$ and since $S^i_\beta$ is correct we know that for all these $k$, $\sigma,\tau,k \nvDash \beta$. Given $a_a > b_a$ we know that $\exists k' \in [b_a,\tau_j].(\rp{k'}{\bot} \in S^i_\alpha$ which means that $\sigma, \tau, k' \nvDash \alpha$. From the semantics of since and the above restrictions, for $\phi$ to be satisfied there must exist a point $p \in [\tau_j-h,b_a]. \sigma, \tau, p \vDash \beta$ and $\forall p' \in [p, \tau_j]. \sigma, \tau, p' \vDash \alpha$. But we already know (since $p' \leq b_a$) that $\exists p' \in [p,\tau_j] \sigma, \tau, p' \nvDash \alpha$, so $\alpha\, \mathcal{S}_{[l,h]}\, \beta$ cannot hold.

%which from the definition of reduce we can see that $\exists k.(\tau_k \in [\tau_j-h,\tau_j] \wedge \rp{k}{\bot} \in S^i_\alpha)$ and $\exists k' < k. \tau_{k'} \in [\tau_j-h,\tau_j-l] \wedge \rp{k}{\beta'} \in S^i_\beta \wedge \beta' \neq \bot$.
\hspace{2em} \textbf{Sub-Subcase} $\phi = \psi'$
This case is trivial, since $\sigma,\tau,j\vDash \psi' \rightarrow \sigma,\tau,j\vDash \psi'$

%%%%%%%%%%%%%%%%%%%%%%%%%%%%%%%%%%%%%%%%%%%%%%%%%%%%%%
%%%% need to remove this one we get going.
%OLD%\begin{lemma}[Correctness of $\mathbf{reduce}(\sigma, \tau, i, \mathbb{S}^i_{\psi}, \rpt{j}{\psi'}{\psi})$]
%OLD%For all positions $i,j$ with $j \leq i$, formulas $\psi$, traces $\sigma$, time series $\tau$, $\psi'$ such that 
%OLD%$\sigma, \tau, j \vDash \psi' \leftrightarrow \sigma, \tau, j, \psi$, and all $S^i_{\phi'} \in \mathbb{S}^i_{\psi}$ correct it is the case that
%OLD%for $\rp{j'}{\phi} \leftarrow \mathbf{reduce}(\sigma, \tau, i, \mathbb{S}^i_{\psi}, \rpt{j}{\psi'}{\psi})$ 
%OLD%1) $\sigma, \tau, j \vDash \psi \leftrightarrow \sigma, \tau, j \vDash \phi$ and
%OLD%2) if $\tau_i - \tau_j \geq \Delta^w(\psi)$ then $\phi \in \{\top, \bot\}$
%OLD%\end{lemma}
%OLD%
%OLD%\noindent Proof is by induction over the formula $\phi'$: \\
%OLD%%% TRUE
%OLD%\noindent \textbf{Case} $\psi' \equiv \top$ \\
%OLD%This case returns the given residue, and since $\sigma, \tau, j \vDash \psi' \leftrightarrow \sigma, \tau, j \vDash \psi$, 
%OLD%property 1 is satisfied and since $\psi' \in \{\top,\bot\}$ property 2 is satisfied.
%OLD%\\ \\ %% FALSE 
%OLD%\noindent \textbf{Case} $\psi' \equiv \bot$ \\
%OLD%This case returns the given residue, and since $\sigma, \tau, j \vDash \psi' \leftrightarrow \sigma, \tau, j \vDash \psi$, 
%OLD%property 1 is satisfied and since $\psi' \in \{\top,\bot\}$ property 2 is satisfied.
%OLD%\\ \\ %% PROP
%OLD%\noindent \textbf{Case} $\psi' \equiv p$ \\
%OLD%In this case reduce returns $\rp{j}{\sigma(p)}$ -- this satisfies property one by the semantics of our logic ($\sigma, \tau, j \vDash p \leftrightarrow \sigma(p)$).
%OLD%Property 2 is trivially satisfied as this always returns $\phi \in \{\top, \bot\}$
%OLD%\\ \\ %% NOT
%OLD%\noindent \textbf{Case} $\psi' \equiv \neg \phi$ \\
%OLD%In this case we first obtain $\psi_1 \leftarrow \mathbf{reduce}(\phi)$ and then obtain $\psi_2 \leftarrow \mathbf{simplify}(\neg \psi_1)$.
%OLD%Since $\mathbf{simplify}$ is correct (Lemma 1) and $\sigma, \tau, j \vDash \psi_1 \leftrightarrow \sigma, \tau, j \vDash \phi$ 
%OLD%	we have $\sigma, \tau, j \vDash \psi_2 \leftrightarrow \sigma, \tau, j \vDash \neg \phi$.
%OLD%
%OLD%Since $\mathbf{reduce}$ is prompt we know that $\psi_1$ is prompt and since $\mathbf{simplify}$ is correct then this case will be prompt (property 2).
%OLD%\\ \\ %% OR
%OLD%\noindent \textbf{Case} $\psi' \equiv \alpha \vee \beta$ \\
%OLD%In this case we first obtain $\psi_a \leftarrow \mathbf{reduce}(\alpha)$ and $\psi_b \leftarrow \mathbf{reduce}(\beta)$.
%OLD%So we have $\sigma, \tau, j \vDash \psi_a \leftrightarrow \sigma, \tau, j \vDash \alpha$ and similarly for $\beta$.
%OLD%Next we recombine these into $\psi_{ab} = \psi_a \vee \psi_b$ and obtain $\phi \leftarrow \mathbf{simplify}(\psi_{ab})$.
%OLD%Since $\mathbf{simplify}$ is correct we know that $\sigma, \tau, j \vDash \phi \leftrightarrow \sigma, \tau, j \vDash \psi_a \vee \psi_b$.
%OLD%
%OLD%Since $\mathbf{reduce}$ is prompt we know that $\psi_a$ and $\psi_b$ are prompt, so $\psi_{ab}$ is also prompt and thus $\phi$ is prompt.
%OLD%\\ \\ %% UNTIL
%OLD%\noindent \textbf{Case} $\psi' \equiv \alpha \mathcal{U}_{[l,h]} \beta$ \\
%OLD%In this case we calculate five values and then use these values to determine the appropriate reduction. There are 4 possible reductions:
%OLD%\begin{enumerate}
%OLD%\item $b_t \leq a_u$ -- returning $\rp{j}{\top}$
%OLD%\item $a_a < b_a$ -- returning $\rp{j}{\bot}$
%OLD%\item $b_n \equiv \top$ -- returning $\rp{j}{\bot}$
%OLD%\item otherwise returning $\rp{j}{\psi'}$
%OLD%\end{enumerate}
%OLD%
%OLD%The first case is for when the formula is true. If $b_t$ is not empty then $\sigma, \tau, b_t \vDash \beta$ and 
%OLD%if $a_u$ is non-empty then 
%OLD%	$\forall n' \in [\tau_j,\tau_j+h].(n' < a_u \rightarrow \sigma, \tau, n' \vDash \alpha)$
%OLD%	so if $b_t \leq a_u$ then this is the truth semantics of Until ($\exists t. \sigma, \tau, t \vDash \beta$ and $\forall t' \in [\tau_j+l,t] \sigma, \tau, t' \vDash \alpha$)
%OLD%
%OLD%The second case is the most complex. In this case we check whether it's possible for $\psi'$ to be true (whether $\alpha$ can be true until the first possible true $\beta$).
%OLD%If $a_a$ is non-empty then $\sigma, \tau, a_a \nvDash \alpha$. If $b_a$ is non-empty then $b_a$ is the minimum position in $[\tau_j+l,\tau_j+h]$ where it's possible that $\beta$ is true, i.e. $\forall k \in [\tau_j+l,\tau_j+h].(k < b_a \rightarrow \sigma, \tau, k \nvDash \beta)$.
%OLD%So if $a_a < b_a$ then there are two cases: 
%OLD%\begin{enumerate}
%OLD%\item $\exists k \in [\tau_j+l,\tau_j+h].(\sigma,\tau,k \vDash \beta)$  -- in this case we know that $k \geq b_a$ and thus $\psi'$ must be false since we already know that $\sigma,\tau,a_a \nvDash \alpha$ and $a_a \in [\tau_j, k]$ so we cannot have $\alpha$ until $\beta$.
%OLD%\item $\not\exists k \in [\tau_j+l,\tau_j+h].(\sigma,\tau,k \vDash \beta)$ -- in this case $\alpha \mathcal{U}_{[l,h]} \beta$ is false, so we can return $\rp{j}{\bot}$
%OLD%\end{enumerate}
%OLD%
%OLD%The third case, if $b_n \equiv \top$ then because $S^i_{\beta}$ is prompt we know that $\not\exists t \in [\tau_j+l,\tau_j+h].(\sigma,\tau,t \vDash \beta)$. 
%OLD%Since this is true we know that $\alpha \mathcal{U}_{[l,h]} \beta$ must be false and so we can return $\rp{j}{\bot}$
%OLD%
%OLD%This last case is trivially correct.
%OLD%
%OLD%For promptness, there are two base cases -- either $\tau_i-\tau_j \geq \Delta^w(\psi)$ (i.e., the residue is past due) or $\tau_i-\tau_j < \Delta^w(\psi)$ (in which case the residue is not due yet, and so any returned value is prompt).
%OLD%
%OLD%So we only need to deal with the past due scenario: There are three possible scenarios:
%OLD%
%OLD%\noindent \textbf{Subcase} \textit{No Beta} \\
%OLD%There does not exist a $k \in [\tau_j+l,\tau_j+h].(\sigma,\tau,k \vDash \beta)$. In this case, $b_n$ will be $\top$ so reduce will return $\rp{j}{\bot}$, satisfying promptness.
%OLD%\\ \\
%OLD%\noindent \textbf{Subcase} \textit{Not alpha until Beta} 
%OLD%$\exists k \in [\tau_j+l,\tau_j+h].(\sigma,\tau,k \vDash \beta)$ but $\not\forall k' \in [\tau_j,k].(\sigma, \tau, k' \vDash \alpha)$. In this case, since there is a $\beta$ within the bounds then $b_a = k$. Since in this case there exists a $k' \in [\tau_j,k]$ where $\sigma,\tau, k' \nvDash \alpha$ then $a_a = k'$. Since we know that $k' < k$, we have $a_a < b_a$ and so reduce will return $\rp{j}{\bot}$.
%OLD%\\ \\
%OLD%\noindent \textbf{Subcase} \textit{Alpha until Beta} 
%OLD%In this case, $\exists k \in [\tau_j+l,\tau_j+h].(\sigma,\tau,k\vDash \beta)$ and $\forall k' \in [\tau_j,k].(\sigma,\tau,k' \vDash \alpha)$. So we have $b_a = k$ and $a_u \geq k$ (since the minimum residue in $[\tau_j,\tau_j+h]$ which is not $\top$ must be after $k$). So we have $b_t \leq a_u$ and so reduce returns $\rp{j}{\top}$.
%OLD%\\ \\ %% SINCE
%OLD%\noindent \textbf{Case} $\psi' \equiv \alpha \mathcal{S}_{[l,h]} \beta$ \\
%OLD%This proof is the same as the Until case
%OLD%
%%%%%%%%%%%%%%%%%%%%%%%%%%%%%%%%%%%%%%%%%%%%%%%%%%%%%%%
%%%%%%%%%%%%%%%%%%%%%%%%%%%%%%%%%%%%%%%%%%%%%%%%%%%%%%%
\pagebreak
\section{Algorithms}
\subsection{Top-level monitoring Algorithm}
Our top level aggressive monitoring algorithm is as follows:

\begin{algorithmic}[1]
\STATE Recognize formulas for which we build structures
\STATE For all recognized formulas $\psi$: $S^{-1}_{\psi} \leftarrow \emptyset$
\STATE $i \leftarrow 0$
\LOOP
\FOR{every $\psi \in \text{temp\_sub}(\varphi)$ in increasing size}
	\STATE $S^i_{\psi} \leftarrow \bigcup\limits_{r \in S^{i-1}_{\psi}} \mathbf{reduce}(\sigma, \tau, i, \mathbb{S}^i_{\psi}, r) \cup \mathbf{reduce}(\sigma, \tau, i, \mathbb{S}^i_{\psi}, \rp{i}{\psi})$
\ENDFOR
\FOR{all $\rp{j}{\bot} \in S^i_{\varphi}$}
\STATE \texttt{Report violation on $\sigma$ at position $j$}
\ENDFOR
\STATE $i \leftarrow i + 1$
\ENDLOOP
\end{algorithmic}


%%% simplify
\begin{align*}
\mathbf{simplify}(\phi) = \left\{
\begin{aligned}
&\top &\text{ if } \phi \equiv \neg \bot \\
&\bot &\text{ if } \phi \equiv \neg \top \\
&\top &\text {if } \phi \equiv \psi_1 \vee \psi_2 \\
& &\text{ and } \\ & & \psi_1 \equiv \top \text{ or } \psi_2 \equiv \top \\
&\phi &\text{ otherwise}
\end{aligned} \right. \\
\end{align*}

%%% temp_sub/need
\begin{align*}
\mathbf{tempSub}(\psi) = \left\lbrace
\begin{aligned}
\{\psi_1\} \cup \{\psi_2\} \cup \mathbf{tempSub}(\psi_1) \\
\cup \mathbf{tempSub}(\psi_2) & \quad \text{ if } \psi \equiv \psi_1 \mathcal{U}_{[l,h]} \psi_2 \text{ or } \psi \equiv \psi_1 \mathcal{S}_{[l,h]} \psi_2 \\
\emptyset & \quad \text{ if } \psi \equiv p \\
\mathbf{tempSub}(\psi_1) \cup \mathbf{tempSub}(\psi_2) & \quad \text{ if } \psi \equiv \psi_1 \vee \psi_2 \\
\mathbf{tempSub}(\psi_1) & \quad \text{ if } \psi \equiv \neg \psi_1
\end{aligned} \right.
\end{align*}
%%% Delta
\begin{align*}
\Delta(\phi) = \left\lbrace
\begin{aligned}
0 & \quad \text{ iff } \psi \equiv \top \\
0 & \quad \text{ iff } \psi \equiv \bot \\
0 & \quad \text{ iff } \psi \equiv p \\
\Delta(\psi) & \quad \text{ iff } \phi \equiv \neg \psi \\
%max(\Delta(\alpha),\Delta(\beta)) & \quad \text{ iff } \phi \equiv \alpha \wedge \beta \\
max(\Delta(\alpha),\Delta(\beta)) & \quad \text{ iff } \phi \equiv \alpha \vee \beta \\
%max(\Delta(\alpha),\Delta(\beta)) & \quad \text{ iff } \phi \equiv \alpha \rightarrow \beta \\
%H + \Delta(\psi) & \quad \text{ iff } \phi \equiv \square_{l,H} \psi \\
%h + \Delta(\psi) & \quad \text{ iff } \phi \equiv \lozenge_{l,h} \psi \\
h + max(\Delta(\alpha),\Delta(\beta)) & \quad \text{ iff } \phi \equiv \alpha\, \mathcal{U}_{[l,h]}\, \beta \\
%\Delta(\psi) & \quad \text{ iff } \phi \equiv \blacksquare_{l,H} \psi \\
%\Delta(\psi) & \quad \text{ iff } \phi \equiv \blacklozenge_{l,h} \psi \\
max(\Delta(\alpha),\Delta(\beta)) & \quad \text{ iff } \phi \equiv \alpha\, \mathcal{S}_{[l,h]}\, \beta \\
\end{aligned} \right. 
\end{align*}

%%% Delta^w
\begin{align*}
\Delta^w(\phi) = \left\lbrace
\begin{aligned}
0 & \quad \text{ iff } \psi \equiv \top \\
0 & \quad \text{ iff } \psi \equiv \bot \\
0 & \quad \text{ iff } \psi \equiv p \\
\Delta^w(\psi) & \quad \text{ iff } \phi \equiv \neg \psi \\
%max(\Delta(\alpha),\Delta(\beta)) & \quad \text{ iff } \phi \equiv \alpha \wedge \beta \\
max(\Delta^w(\alpha),\Delta^w(\beta)) & \quad \text{ iff } \phi \equiv \alpha \vee \beta \\
%max(\Delta(\alpha),\Delta(\beta)) & \quad \text{ iff } \phi \equiv \alpha \rightarrow \beta \\
%H + \Delta(\psi) & \quad \text{ iff } \phi \equiv \square_{l,H} \psi \\
%h + \Delta(\psi) & \quad \text{ iff } \phi \equiv \lozenge_{l,h} \psi \\
h + max(\Delta^w(\alpha),\Delta^w(\beta)) & \quad \text{ iff } \phi \equiv \alpha\, \mathcal{U}_{[l,h]}\, \beta \\
%\Delta(\psi) & \quad \text{ iff } \phi \equiv \blacksquare_{l,H} \psi \\
%\Delta(\psi) & \quad \text{ iff } \phi \equiv \blacklozenge_{l,h} \psi \\
max(\Delta^w(\alpha),\Delta^w(\beta)) - l & \quad \text{ iff } \phi \equiv \alpha\, \mathcal{S}_{[l,h]}\, \beta \\
\end{aligned} \right. 
\end{align*}

%\displaybreak
\subsubsection*{Reduce}
The reduce function takes a residue and returns the reduced residue:

%\begin{align*}

%%% true/false
\begin{align*}
\mathbf{reduce}(\sigma,\tau, i,\pi,\rp{j}{\top}) &= \rp{j}{\top} \\
\mathbf{reduce}(\sigma,\tau, i,\pi,\rp{j}{\bot}) &= \rp{j}{\bot} \\
\mathbf{reduce}(\sigma,\tau, i,\pi,\rp{j}{p}) &= \rp{j}{\sigma_i(p)} \\
\end{align*}
%%% NOT
\begin{align*}
\mathbf{reduce}(\sigma,\tau, i,\pi,\rp{j}{\neg \psi}) = \left\{
\begin{aligned}
\text{Let } &\psi' \leftarrow \mathbf{reduce}(\sigma, \tau, i, \pi, \rp{j}{\psi}) \\
&\psi'' \leftarrow \mathbf{simplify}(\neg \psi') \\
&\textbf{return } \rp{j}{\psi''} \\
\end{aligned} \right. \\
\end{align*}

%%% OR
\begin{align*}
\mathbf{reduce}(\sigma,\tau, i,\pi,\rp{j}{\alpha \vee \beta} = \left\{
\begin{aligned}
\text{Let } &\alpha' \leftarrow \mathbf{reduce}(\sigma, \tau, i, \pi, \rp{j}{\alpha}) \\
&\beta' \leftarrow \mathbf{reduce}(\sigma, \tau, i, \pi, \rp{j}{\beta}) \\
&\phi' \leftarrow \mathbf{simplify}(\alpha' \vee \beta') \\
&\textbf{return } \rp{j}{\phi'} \\
\end{aligned} \right. \\
\end{align*}

%%% TAU UNTIL
\begin{align*}
\mathbf{reduce}(\sigma,\tau, i,\pi,\rp{j}{\alpha\, \mathcal{U}_{[l,h]}\, \beta}) = \left\{
\begin{aligned}
&\text{let } a_a \leftarrow min(\{k | \tau_k - \tau_j \in [0, h] \wedge \rp{k}{\bot} \in S^i_{\alpha} \}, i+1)  \\
&a_u \leftarrow min(\{k | \tau_k - \tau_j \in [0, h] \wedge  \rp{k}{\psi'} \in S^i_{\alpha} \wedge \psi' \neq \top \}, i) \\
&b_a \leftarrow min(\{k | \tau_k - \tau_j \in [l, h] \wedge \rp{k}{\psi'} \in S^i_{\beta} \wedge \psi' \neq \bot \}, i)  \\
&b_t \leftarrow min(\{k | \tau_k - \tau_j \in [l, h] \wedge \rp{k}{\top} \in S^i_{\beta} \}) \\
&b_n \leftarrow \top \text{ if } (\tau_i - \Delta^w(\psi) \geq \tau_j) \wedge \forall k.(\tau_k - \tau_i \in [l,h] \rightarrow \rp{k}{\bot} \in S^i_{\beta}) \\
&\text{if }b_t \neq \emptyset \wedge b_t \leq a_u \\
& \quad\mathbf{return} \rp{j}{\top} \\
&\text{else if } a_a < b_a \text{ or } b_n = \top\\ & \quad\mathbf{return} \rp{j}{\bot} \\
&\text{else} \\
& \quad\mathbf{return} \rp{j}{\alpha\, \mathcal{U}_{[l,h]}\, \beta}
\end{aligned} \right. \\
\end{align*}

%%% TAU SINCE

\begin{align*}
\mathbf{reduce}(\sigma,\tau, i,\mathbb{S}^i_{\phi} ,\rp{j}{\alpha\, \mathcal{S}_{[l,h]}\, \beta}) = \left\{
\begin{aligned}
&\text{let } a_a \leftarrow max(\{k | \tau_k \in [\tau_j-h, \tau_j] \wedge \rp{k}{\bot} \in S^i_\alpha \}) \\ 
& a_s \leftarrow min(\{k | \tau_k \in [\tau_j-h, \tau_j]  \\
& \quad \quad \wedge \forall k' \in [\tau_k,\tau_j].(\rp{k'}{\alpha'} \in S^i_\alpha \wedge \alpha' = \top) \}, j+1) \\
%
& b_a \leftarrow max(\{k | \tau_k \in [\tau_j-h, \tau_j-l] \wedge \rp{k}{\beta'} \in S^i_\beta \wedge \beta' \neq \bot\}) \\ 
&b_t \leftarrow max(\{k | \tau_k \in [\tau_j-h, \tau_j-l] \wedge \rp{k}{\top} \in S^i_{\beta} \}) \\
&b_n \leftarrow \top \text{ if } (\tau_i - \Delta^w(\psi) \geq \tau_j) \wedge \forall k.(\tau_k \in [\tau_j-l,\tau_j-h] \rightarrow \rp{k}{\bot} \in S^i_{\beta}) \\
&\text{if } b_t \neq \emptyset \wedge a_s \leq b_t + 1\\
& \quad\mathbf{return} \rp{j}{\top} \\
&\text{else if } (a_a \neq \emptyset \wedge b_a \neq \emptyset) \wedge a_a > b_a \text{ or } b_n = \top\\ & \quad\mathbf{return} \rp{j}{\bot} \\
&\text{else} \\
& \quad\mathbf{return} \rp{j}{\alpha\, \mathcal{S}_{[l,h]}\, \beta}
\end{aligned} \right. \\
\end{align*}

\end{document}



%%%% old stuff, don't really need but keeping around for now...
\section{Soundness/Completeness Proofs}

The soundness and completeness of our top-level algorithm ultimately depends on the soundness and completeness of the reduce algorithm. So first, we show that reduce is sound and complete.

\begin{thm}
The function $\hat{r}^{\phi}_{j} \leftarrow reduce(\sigma_i, \tau_i, \mathbb{R}^i, \res{j}{\phi})$ is sound, complete, and prompt when $\mathbb{R}^i, \res{j}{\phi}$ are sound, complete, and prompt.
\end{thm}
\noindent We will show that the properties hold over the form of $\phi$: \\
\\ \noindent \textbf{Case} $\phi \equiv \top/\bot$: \\
reduce is trivially sound, complete and prompt since we do not change the residue.

\noindent \textbf{Case} $\phi \equiv p$ \\
reduce is trivially sound, from the semantics we know $\sigma, \tau, j \vDash p \leftrightarrow \sigma, \tau, j \vDash \sigma_j(p)$. Promptness is trivial since reduce will always return the reduced truth value of a proposition.  For completeness we know that $\sigma, \tau, j \vDash p \rightarrow \sigma, \tau, j \vDash \sigma_j(p)$ and $\sigma, \tau, j \nvDash p \rightarrow \sigma, \tau, j \nvDash \sigma_j (p)$, so replacing $p$ with $\sigma_j(p)$ is complete.
\\ \\ 
\noindent \textbf{Case} $\phi \equiv \neg \psi$ \\
reduce is trivially sound...
\\ \\
\noindent \textbf{Case} $\phi \equiv \psi_1 \vee \psi_2$ \\
...
\\ \\
\noindent \textbf{Case} $\phi \equiv \psi_1 \mathcal{U}_{[l,h]} \psi_2$: \\
Soundness is simple in this case, since reduce either returns $\res{j}{\phi}$ which is trivially sound, or it returns the semantically correct truth value ($\top/\bot$). Reduce will only return $\top$ if $\exists k \in [j+l,j+h]$ s.t. $\res{k}{\psi_2} = \top$ and $\forall k' \in [j,e]$ $\res{k'}{\psi_1} = \top$ (if any $\psi_1$ residue is $\bot$ then $\bot$ will be returned, and if any are not yet evaluated then unfinished is set and the original residue is returned). This is the semantics of $\mathcal{U}$, so we know that a returned $\top/\bot \rightarrow \sigma, \tau, j \vDash \phi$. Thus reduce is sound for $\mathcal{U}$.

Completeness is trivial because the only possible reduction is a sound $\top/\bot$ value or the original $\phi$. So for all $\res{j}{\psi} \in S^i_{\psi_1 \mathcal{U}_{[l,h]} \psi_2}$ we have $\res{\psi}{j} \equiv \rp{j}{\psi_1 \mathcal{U}_{[l,h]} \psi_2}$ or $\res{\psi}{j} \equiv \top/\bot$. Since reduce is sound, if the residue is $\top/\bot$ we know that the residue value is correct ($\top$ if $\sigma, \tau, j \vDash \phi$ and $\bot$ otherwise) and if a residue $\res{\phi}{j} \equiv \rp{j}{\phi}$ it is trivially complete.

Reduce is prompt $\mathbf{reduce}$


\section{Description}
Our runtime monitoring algorithm \agmon takes as input a well-formed bmtl formula $\varphi$, monitors the system trace as it grows, builds history structures for temporal subformula of $\varphi$ and reports violations of $\varphi$ as soon as they are detected.

We summarize the relevant functions below: \\
$\agmon(\varphi)$ is our top-level monitor function. \\
$\mathbf{incr}(\sigma_i, \tau, i, \hat{\mathcal{R}}^i, \psi)$ updates the history structure $S^i_{\psi}$ based on the new trace entry $\sigma_i$ \\ 
$\mathbf{reduce}(\sigma_i,\tau, i, \mathcal{R}^i, \res{\psi}{j})$ takes a residue $\res{\psi}{j}$ and returns a reduced residue based on the history $\mathcal{R}^i$ and current state $\sigma_i, \tau_i$. 
