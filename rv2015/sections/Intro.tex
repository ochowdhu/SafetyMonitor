\section{Introduction}
Embedded systems, from home appliances to automobiles, are becoming increasingly complex due to the addition of new advanced features. 
It is paramount for these embedded systems, specifically the safety-critical (SC) systems (\eg, autonomous vehicles, aircraft flight control),  
to be formally verified to avoid catastrophic consequences  of malfunctioning (\eg, loss of life, loss of properties) due to implementation errors or 
design flaws. Static verification techniques like model checking and theorem provers are applicable to obtain provable guarantees about the SC systems' 
correctness. However, model checking suffers from the state-space explosion problem whereas theorem provers require significant manual intervention.  
Runtime verification (RV) technique is a promising alternative to its static counterpart for checking system safety and correctness in the face of 
increasing design complexity. 
In RV,  a runtime monitor (RM) observes the execution of the system in question and checks for violation of some well-defined properties. 
When the RM detects a violation, it can notify a command module which then attempts to recover from the violation. \emph{In this paper, we develop a 
RM system that monitors an autonomous research vehicle (ARV) system and describe our experience with it}.

The ARV system we consider are built by system integrators 
utilizing multiple vendors and commercial-off-the-shelf (COTS) components, some of which can be 
viewed as black box components for the integrator. 
These systems are often  hard real-time systems which lead to more constraints on 
system monitoring \cite{Goodloe2010}. This type of architecture is incompatible with many 
existing runtime monitoring techniques, which often require program or system instrumentation \cite{} to obtain 
the relevant events or policy-constructs (\eg, propositions) to check for violation. 
Instrumenting systems without the access of source code  is more difficult, 
and even when the source  is available there are risks of affecting the timing and 
correctness of the target system when instrumented.

Instead of instrumentation, we obtain the relevant information for monitoring the ARV system through passive observation 
of the broadcast buses. The broadcast buses in the ARV system (\eg, controller area network (\textit{CAN}) in ground vehicles) 
 contain a large amount of useful system information relevant to monitoring 
within the messages sent between different system components. However, before we can start 
monitoring the ARV system, we need a component, which we call the \textsf{SF Map}, that 
observes messages going through the bus, decodes them to obtain policy-constructs 
relevant to monitoring, and feed them to the RM. We want to emphasize that depending on the granularity 
of the information available in the messages and the atomic policy-constructs, developing the \textsf{SF Map} 
poses a significant challenge. 

To obtain the relevant safety requirements and invariants for monitoring the ARV system, we consulted the safety 
manual of the ARV system. We observed that many desired system properties are timing related, so using an explicit-time 
based specification language for expressing these properties is helpful. System requirements such as ``\emph{the system must 
perform action $a$ within $t$ seconds of event $e$}'' are common. For efficient monitoring, we use a fragment of propositional 
metric temporal logic (MTL)\cite{MTL} in which the bound associated with the future temporal operators can only be finite. 

%is a commonly used logic for specifying these types of properties, and a bounded-future fragment of MTL can be used to ensure efficient monitoring of useful system properties.
% talk about explicit time? real time? something in that vein

%% third paragraph -- contributions
In this paper we present a real-time embedded monitor for safety-critical embedded systems with black-box COTS components (such as automobiles). Our monitoring algorithm \monitor is an dynamic programming based iterative algorithm which utilizes formula reduction (essentially rewriting) and history structures to system traces obtained from a target broadcast bus against a given safety policy. We have implemented this algorithm on an inexpensive embedded platform and present a case study using the monitor to perform real-time checking of replayed CAN logs from the robustness testing of an autonomous vehicle.

%% 15 pages in this format is so short...
%Due to space restrictions, we defer the correctness proof of \monitor and other details to a technical report \cite{TechReport}.

