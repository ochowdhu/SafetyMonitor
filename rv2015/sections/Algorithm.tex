%%%%% Algorithm section

\section{Monitoring Algorithm}
%% Why our algorithm...
%%% MTL due to explicit time restrictions
%%% need real-time finite trace checking

%In order to
For checking whether the given ARV system  adheres to its specification,
%distributed embedded systems
we need an algorithm which %can %continuously
incrementally checks explicit time specifications (\ie, propositional metric-time temporal logic~\cite{Koymans1990})
over finite, timed system traces (Every trace position of a \emph{timed trace} has an explicit time associated with it).
This has led to our algorithm \monitor which is an iterative monitoring algorithm based on formula rewriting and summarizing the relevant history of the trace
in \emph{history-structures}. To detect violations early, \monitor, eagerly checks whether it can reduce subformulas of the original formula to boolean value
%either true/false
using formula simplifications (\eg, $a\wedge \mathit{false} \equiv \mathit{false}$).
Many of the existing algorithms for evaluating formulas like $\eventually_{[l,h]} a \vee b$
(read, either $b$ is true or sometimes in the future $a$ is true such that the time difference
between the evaluation state and the future state in which $a$ is true, $t_d$, is within the
bound $[l,h]$) wait enough time so that $\eventually_{[l,h]} a$ can be fully evaluated.
\monitor however tries to eagerly evaluate both $\eventually_{[l,h]} a$ and $b$ and see
whether it can reduce the whole formula to boolean value.
For another eagerness checking example,
let us assume we are checking the property $a\until_{[l, h]}b$ (read, the formula is true at trace position $i$ if there is a trace position
$j$ in which $b$ holds such that $j\geq i$ and the time difference between position $i$ and $j$ is in the range $[l,h]$
and for all trace positions $k$ such that $i\leq k < j$ the formula $a$ holds) at trace position $i$.
During checking if we can find a trace position $k > i$ for which $b$ and $a$ are both false then we can evaluate the formula to be
false without waiting for a trace position in which $b$ is true. Nesting of  temporal formulas can make the eager check nontrivial.
We want to emphasize that \monitor optimistically
checks for violations and hence we could possible have a trace in which each formula can only be evaluated in
the last possible trace position in which case our algorithm will behave in the exact same way as the current non-eager
algorithms modulo the extra computation \monitor performs for eager checking.
%either true or false.
%
%To constantly check that the target system is adhering to its specification, we must constantly check the specification against the current system trace. To constantly check the system efficiently, \monitor is an iterative algorithm, performing additional checking as each new trace step arrives.

\subsection{Specification Logic}
Many safety specification rules for ARV-like systems require explicit time bounds to ensure timely behavior,
so a specification language with explicit time bounds is important.
For instance,
%Examples of rules that utilize explicit time bounds include: %
``\natlangrule{cruise control shall disengage for 250ms within 500ms of the brake pedal being depressed}''
($\pred{brakeDepress} \rightarrow \eventually_{[0, 500]} \henceforth_{[0,250]} \neg \pred{CruiseEng}$).
%,
%\natlangrule{the cruise control mode cannot transition directly from Off to Engaged within 250ms} ($\neg (\pred{CruiseEngaged} \wedge \yesterday_{[0,250]} \pred{CruiseOff})$),
%and \natlangrule{the vehicle should not be within 1m of the current lane edge for 1s while lane centering is enabled} ($\neg \historically_{[0, 1000]}(\pred{LaneCentering} \wedge \pred{distToLaneEdgeLT1m})$).
Our safety specification language for the ARV system, which we call \planguage, is a future-bounded,
discrete time, propositional metric temporal logic (MTL \cite{Koymans1990}).
%%past temporal operators may have infinite bounds but
%where
%future temporal operators must have finite bounds.
%(\ie, $l$ and $h$ are both finite).
The syntax of \planguage is as follows:

\vspace*{1pt}
\(
\begin{array}{ccc}
\policy & ::=  & \true \mid \pred{p} \mid \neg \policy \mid \policy_1\vee \policy_2\mid
\policy_1\since_{\interval}\policy_2 \mid \policy_1\until_{\interval}\policy_2\mid\yesterday_{\interval}\policy\mid \tomorrow_{\interval}\policy
\end{array}
\) \\
\textit{Syntax.}
\planguage has logical true (\ie, \true),
 propositions $\pred{p}$, logical connectives (\ie, $\neg, \vee$),
past temporal operators \emph{since} and \emph{yesterday} $(\since, \yesterday)$,
and future temporal operators \emph{until} and \emph{next} $(\until, \tomorrow)$.
There is a bound $\mathbb{I}$ of form $[l,h]$ ($l\leq h$ and $l,h\in\mathbb{N}\cup\infty$)
associated with each temporal operator.
Note that the bound  $[l,h]$ associated with the future temporal operators must be finite.
Specification propositions \pred{p} come from a finite set of propositions provided in
the system trace by the \sfmap.
These propositions are derived from the observable system state and represent specific system properties, for instance,
proposition $\pred{speedLT40mph}$ could describe whether the vehicle speed is less than 40mph.
We use \policy, $\phi$, $\alpha$, and $\beta$ (possibly with subscripts) to denote valid \planguage formulas.
%


\textit{Semantics. }
\planguage formulas are interpreted over time-stamped \emph{traces}. A trace $\sigma$ is a sequence of states, each of which maps all propositions in \sfmap, to either \true or \false. We denote the $i^{th}$ position of the trace with $\sigma_i$ where $i\in\mathbb{N}$. Moreover, each $\sigma_i$ has an associated time stamp denoted by $\tau_i$ where $\tau_i\in\mathbb{N}$.
We denote the sequence of time stamps with $\tau$. For all $i, j\in\mathbb{N}$ such that $i < j$, we require $\tau_i < \tau_j$. For a given trace $\sigma$ and time stamp sequence $\tau$, we write $\sigma, \tau, i\models\policy$ to denote that the formula \policy is true with respect to the $i^{th}$ position of $\sigma$ and $\tau$.
The semantics of \planguage future bounded MTL is standard, see for instance, \cite{Basin2008}.
Each  property \policy has an implicit unbounded \henceforth future operator ($\henceforth\policy$ signifies that
\policy is true in all future trace positions including the current trace position) at the top-level and we handle it by
checking whether \policy is true in each trace position.
% is evaluated over the trace at every trace step (\ie, there is an implicit unbounded always over all top-level specification formulas.

\[
\tempSub(\policy) = \begin{cases}
\emptyset & \mbox{ if } \policy \equiv p\\
\{\alpha\} \cup \{\beta\} \cup \tempSub(\alpha) \cup \tempSub(\beta) & \mbox{ if } \policy\equiv\alpha\until_\mathbb{I}\beta|\alpha\since_\mathbb{I}\beta\\
\tempSub(\alpha) \cup \tempSub(\beta) & \mbox{ if } \policy\equiv\alpha\vee\beta\\
\tempSub(\alpha) & \mbox{ if }\policy\equiv\neg\alpha
\end{cases}
\]
% \textit{Auxiliary notions.}
Before we go any further, we introduce the readers with some auxiliary notions which will be
necessary to understand our algorithm \monitor. We first define ``\emph{residual formulas}'' or, just ``\emph{residues}''.
Given a formula \policy, we call another formula \policyv as \policy's residual, if we obtain \policyv after evaluating \policy with respect to the current information of the trace.
Note that a formula residue might not be a truth value if the formula could not conclusively be reduced given the current trace state (e.g, if future state is required to determine the truth value).
A residue $r^j_{\policy}$ is a tagged pair $\rpt{j}{\policyv}{\policy}$ where $j$ is a position in the trace in which we intend to evaluate $\policy$ (the original formula) and $\policyv$ is the current residual formula. We use these residues to efficiently hold trace history for evaluating temporal formulas.
%
%Our monitoring algorithm utilizes \emph{residues}, which are partially reduced (\ie, rewritten) policy formulas representing the remaining portion of a formula which could not be fully evaluated given the current trace. A residue $r^j_{\policy}$ is a tagged pair $\rpt{j}{\policyv}{\policy}$ where $j$ is a position in the trace, $\policy$ is the original residue formula and $\policyv$ is the residual formula. We use these residues to efficiently hold policy history for future time formulas which cannot be evaluated due to incomplete information.
The next notion we introduce is of ``\emph{wait delay}''. It is a function \wdelay that takes as input a formula \policy and
$\wdelay(\policy)$ returns
%Policy formulas have a wait delay $\wdelay(\policy)$ which defines
an upper bound on the time one has to wait before they can evaluate \policy with certainty.
%duration necessary to guarantee complete information to evaluate the formula.
For past- and present-time formulas \policyv, $\wdelay(\policyv)=0$.
%,
%as \policyv can be evaluated currently.
%since all trace steps necessary to evaluate them have already been seen by the monitor.
Future time formulas have a delay based on the interval of the future operator
(\eg, $\wdelay(\LTLdiamond_{[0,3]} \pred{p}) = 3$). The length of a formula $\policy$, denoted $|\policy|$, returns the total number of subformulas of \policy.
%, (\ie, the number of nodes in the policy AST).
The function \tempSub takes as input a formula \policy, and returns all the temporal subformulas \policyv of \policy and strict subformulas of \policyv.



%To evaluate an \planguage formula \policy, we may need to save a limited amount of
%past evaluation results for some temporal subformulas of \policy. We call this the \emph{history}.
%%history state of child policies of temporal subformula within a policy.
%For example, given the formula $\pred{ACCCancelReq} \rightarrow \eventually_{\interval} \pred{ACCOff} \vee (\pred{ACCOn} \since_{\interval} \pred{ACCCancelReq})$, $\pred{ACCOff}$ is a child policy of the temporal formula $\eventually_{\interval} \pred{ACCOff}$ and $\pred{ACCOn}$ and $\pred{ACCCancelReq}$ are children of the temporal formula $\pred{ACCOn} \since_{\interval} \pred{ACCCancelReq}$.
%The monitor must store some history of these child policies in order to evaluate the parent policy.
%The operation $\tempSub(\varphi)$ identifies all the children of temporal subformula of a policy $\varphi$.
%%% might want to do a more abstract example to show recursiveness
%%That is, for $\alpha\, \mathcal{U}_{[l,h]}\, \beta$ we need to save the history of $\alpha$ and $\beta$ (and if either of those are also a temporal formula then we need their history as well).


%% maybe need formula length, storage delay, simplify\dots


\subsection{\monitor Algorithm}
Our runtime monitoring algorithm \monitor takes as input an \planguage formula $\policy$ and monitors a growing trace, building history structures and reporting the specification violations as soon as they are detected. We summarize the relevant algorithm functions below:

\begin{description}
\item[$\monitor{}(\policy)$] is the top-level function.
\item[$\reduce(\sigma_i, \tau_i, \histSt{i}, \rpt{i}{\policy}{\policy})$] reduces the given residue based on the current state $(\sigma_i,\tau_i)$ and the history $\histSt{i}$.
\item[$\tempSub(\policy)$] identifies the subformulas which require a history structure to evaluate the formula $\policy$.
\item[$\incrS(\histst{i-1}, \histSt{i}, \sigma_i, \tau_i, i)$] updates the history structure $\histst{i-1}$ to step $i$ given the current trace and history state.
\end{description}




\textbf{Top-level monitoring algorithm.}
The top-level monitoring algorithm \monitor is a sampling-based periodic monitor which uses history structures to store trace state for evaluating temporal subformulas.
\emph{History structures} are lists of residues along with past-time markers for evaluating infinite past-time formulas.
The algorithm checks the given formula $\policy$ periodically at every trace sample step.
When the fomula cannot be decided at a given step (\eg, it requires future state to evaluate), the remaining formula residue is saved in a history structure for evaluation in future steps when the state will be available.
The history structure for formula $\policyv$ at trace step $i$ is denoted $\histst[\policyv]{i}$.
We use $\histSt[\policy]{i}$ to denote the set of history structures for all temporal subformula of $\policy$, \ie,
$\histSt[\policy]{i} = \bigcup_{\policyv \in \tempSub(\policy)} \histst[\policyv]{i}$.


The high level algorithm \monitor is shown in Figure \ref{fig:algorithm}.
%
First, all the necessary history structures $\histst[\policyv]{ }$ are identified using $\tempSub(\policy)$ and initialized.
Once these structures are identified, the monitoring loop begins.
%
In each step, all the history structures are updated with the new trace step.
This is done in increasing formula size since larger formula can depend on the history of smaller formula (which may be their subformula).
%
Each structure is updated using $\incrS(\histst[\policyv]{i-1},\histSt[\policyv]{i}, \sigma_i, \tau_i, i)$ which adds a residue for the current trace step to the structure and reduces all the contained residues with the new step state.
Then, the same procedure is performed for the top level formula that is being monitored -- the formula's structure is updated with $\incrS(\histst{i-1},\histSt{i},\sigma_i,\tau_i, i)$.
Once updated, this structure contains the evaluation of the top-level formula. The algorithm reports any identified formula violations (\ie, any $\false$ residues) before continuing to the next trace step.
%
We note that due to the recursive nature of the monitoring algorithm, the top-level formula is treated exactly the same as any temporal subformula would be (which follows from the fact that the top-level formula contains an implicit \emph{always} $\henceforth$).
The history structure updates for the top-level formula are separated in the algorithm description for clarity only.
The only difference between the top-level formula and other temporal subformula is that violations are reported for the top-level formula.




% \begin{figure}
% \begin{align}
% \tempSub(\phi) &= \emptyset &\text{if } \phi \equiv p \\
% 			   &= \{\alpha\} \cup \{\beta\} \cup \tempSub(\alpha) \cup \tempSub(\beta) &\text{if } \phi \equiv \alpha \until \beta \\
% 			   &				&\text{or } \phi \equiv \alpha \since \beta \\
% 			   &= \tempSub(\alpha) \cup \tempSub(\beta) &\text{if } \phi \equiv \alpha \vee \beta \\
% 			   &= \tempSub(\alpha) &\text{if } \phi \equiv \neg \alpha \\
% \end{align}
% \end{figure}

\begin{figure}[t]
\begin{algorithmic}[1]
%\STATE Recognize formulas for which we build structures
\STATE For all recognized formulas $\policyv \in \tempSub(\policy)$: $\histst[\policyv]{-1} \leftarrow \emptyset$
\STATE $i \leftarrow 0$
\LOOP
\STATE Obtain next trace step $(\sigma_i, \tau_i)$
\FOR{every $\policyv \in \tempSub(\policy)$ in increasing size}
	\STATE $\histst[\policyv]{i} \leftarrow \incrS(\histst[\policyv]{i-1}, \histSt[\policyv]{i}, \sigma_i, \tau_i, i)$
\ENDFOR
\STATE $\histst{i} \leftarrow \incrS(\histst{i-1}, \histSt{i}, \sigma_i, \tau_i, i)$
%\FOR{all $\rp{j}{\bot} \in S^i_{\varphi}$}
\FOR{all $\rp{j}{\false} \in \histst{i}$}
\STATE \texttt{Report violation on $\sigma$ at position $j$}
\ENDFOR
\STATE $i \leftarrow i + 1$
\ENDLOOP
\end{algorithmic}
\caption{\monitor Algorithm}\label{fig:algorithm}
\end{figure}

\textbf{Reducing Residues.}
\monitor works primarily by reducing formula residues down to truth values. Residues are reduced by the $\reduce(\sigma_i, \tau_i, \histSt{i}, \rpt{j}{\policyv}{\policy})$ function, which uses the current state ($\sigma_i,\tau_i$) and the stored history in $\histSt{i}$ to rewrite the formula $\policyv$ to a reduced form, either a truth value or a new formula which will evaluate to the same truth value as the original. For past or present-time formulas, $\reduce()$ is able to return a truth value residue since all the necessary information to decide the formula is available in the history and current state. Future-time policies may be fully-reducible if enough state information is available. If a future-time formula cannot be reduced to a truth value, it is returned as a reduced (potentially unchanged) residue.

% reduce for until
For residues whose formula is an \emph{until} formula $\alpha \until_{[l,h]} \beta$, the history structures $\histst[\alpha]{i}$ and $\histst[\beta]{i}$ are used to reduce the formula.
If the formula can be evaluated conclusively then the truth value is returned, otherwise the residue is returned unchanged.
Figure \ref{fig:until} shows the reduction algorithm for \emph{until} temporal formula. Reducing \emph{since} formulas is essentially the same except with reversed minimum/maximums and past time bounds.

%%% TAU UNTIL
\begin{figure}
\small
\begin{align*}
\mathbf{reduce}(\sigma_i,\tau, i,\mathbb{S}^i_{\alpha\, \mathcal{U}_{[l,h]}\, \beta} ,\rp{j}{\alpha\, \mathcal{U}_{[l,h]}\, \beta}) = \left\{
\begin{aligned}
&\text{let } a_a \leftarrow min(\{k | \tau_j \leq \tau_k \leq \tau_j+h  \wedge \rp{k}{\bot} \in S^i_\alpha \},i) \\
% a_u
%& a_u \leftarrow max({k| \tau_j \leq \tau_k \leq \tau_j+h \wedge \rp{k}{\alpha'} \in S^i_\alpha \wedge \alpha' \not\equiv \top},i) \\
& a_u \leftarrow max(\{k| \tau_k \in [\tau_j,\tau_j+h] \\
& \quad \quad \quad \wedge \forall k' \in [j,k-1].(\rp{k'}{\alpha'} \in S^i_\alpha \wedge \alpha' \equiv \top\},i) \\
% b_a
& b_a \leftarrow min(\{k | \tau_j+l \leq \tau_k \leq \tau_j+h \wedge \rp{k}{\beta'} \in S^i_\beta \wedge \beta' \neq \bot\}) \\
% b_t
&b_t \leftarrow min(\{k | \tau_j+l \leq \tau_k \leq \tau_j+h \wedge \rp{k}{\top} \in S^i_{\beta} \}) \\
% b_n
&b_n \leftarrow \top \text{ if } (\tau_i - \tau_j \geq \wdelay(\psi)) \\
& \quad \quad \quad \wedge \forall k.(\tau_j+l \leq \tau_k \leq \tau_j+h). \rp{k}{\bot} \in S^i_{\beta} \\
&\text{if } b_t \neq \emptyset \wedge a_u \geq b_t \\
& \quad\mathbf{return} \rp{j}{\top} \\
&\text{else if } (b_a \neq \emptyset \wedge a_a < b_a) \text{ or } b_n = \top\\ & \quad\mathbf{return} \rp{j}{\bot} \\
&\text{else} \\
& \quad\mathbf{return} \rp{j}{\alpha\, \mathcal{U}_{[l,h]}\, \beta}
\end{aligned} \right. \\
\end{align*}
%\normalsize
\caption{Definition of \reduce for \emph{until} formulas \label{fig:until}}
\end{figure}

The \reduce function for $\emph{until}$ formulas uses marker values to evaluate the semantics of $\emph{until}$. \reduce calculates five marker values: $a_a$ is the earliest step within the time interval where $\alpha$ is known false. $a_u$ is the latest step within the interval that $\alpha \until_{[l,h]} \beta$ would be true if $\beta$ were true at that step. $b_a$ is the earliest step within the interval at which $\beta$ is not conclusively false, and $b_t$ is the earliest step within the interval at which $\beta$ is conclusively true. $b_n$ holds whether the current step $i$ is later than the wait delay and all $\beta$ values within the interval are false. With these marker variables, reduce can directly check the semantics of \emph{until}, and either return the correct value or the unchanged residue if the semantics are not conclusive with the current history. Reducing \emph{since} formulas works in the same way (using the same marker values) adjusted to past time intervals and utilizing the unbounded past time history values.


\textbf{Incrementing History Structures.}
To evaluate past and future-time policies, we must correctly store trace history which can be looked up during a residue reduction.
We store the trace history of a formula $\policyv$ in a history structure $\histst[\policyv]{}$.
This history structure contains a list of residues for the number of steps required to evaluate the top-level formula. History structures are incremented by the function
$\incrS(\histst[\policyv]{i-1}, \histSt[\policyv]{i}, \sigma_i, \tau_i, i) = (\bigcup_{r \in \histst[\policyv]{i-1}} \reduce(\sigma_i, \tau_i, \histst[\policyv]{i}, r)) \cup \reduce(\sigma_i, \tau_i, \histst[\policyv]{i}, \rp{i}{\policyv})$
%
This function takes the previous step's history structure $\histst[\policyv]{i-1}$ and the current state
%($\sigma_i,\tau_i$, and the updated smaller history structures $\histSt[\policyv]{i}$)
and performs two actions:
%\begin{enumerate}
	1) Adds a residue for the current step $i$ to $\histst[\policyv]{i-1}$ and
	2) Reduces all residues contained in $\histst[\policyv]{i-1}$ with the current state.
%\end{enumerate}
%Together, these two actions leave an updated history structure $\histst[\policyv]{i}$ which has updated history information for all the required steps.


\subsection{Algorithm Properties}
There are two important properties of \monitor which need to be shown. First, \emph{correctness} states that the algorithm's results are correct.
That is, that if \monitor reports a property violation, the trace really did violate the property. 
Second, \emph{promptness} requires that the algorithm provide a decision for the given property in a timely fashion (\ie, with $t$ such that $t\leq \wdelay(\policy)$).
%at any given time $t$ within a finite bound, $\wdelay(\policy)$.
Promptness precisely requires that the algorithm decides satisfaction of the given property at trace position $i$ as soon as there is another trace position $j$ available such that $j\geq i$ and $\tau_j-\tau_i \geq \wdelay(\policy)$.
Note that it is possible to define a stronger notion of promptness, \emph{strong promptness}, which ensures that the runtime monitor makes a judgement about the satisfaction of a property as soon as it is possible to do so. In this discussion we however we only consider promptness.
%guaranteed to be possible to decide.
%
%Besides correctness guarantees, our algorithm is also \emph{eager}. An eager algorithm reports satisfaction as soon as the trace is an informative prefix \cite{Kupferman2001} for the policy.
%Many temporal policies can be evaluated before the promptness delay, providing early detection and extra time for recovery actions compared to other monitoring algorithms which wait until the promptness delay to ensure they can check the policy.
%
The following theorem states that \monitor is correct and prompt. It requires  the history structures $\histSt{i}$ to be consistent at $i$ analogous to the trace $\sigma,\tau$.
This means that the history structures contain correct history of the trace till step $i$. %The algorithm provides this consistency itself if run iteratively from step $0$.

\begin{theorem}[Correctness and Promptness of \monitor]
For all $i \in \mathbb{N}$, all formula $\varphi$, all time stamp sequences $\tau$ and all traces $\sigma$ it is the case that (1) if $\rp{j}{\false} \in \histst{i}$ then $\sigma, \tau, j \nvDash \policy$ and if $\rp{j}{\true} \in \histst{i}$ then $\sigma, \tau, j \vDash \policy$ (Correctness) and (2) if $\tau_i - \tau_j \geq \wdelay(\policy)$ then if $\sigma, \tau, j \nvDash \policy$ then $\rp{j}{\false} \in \histst{i}$ and if $\sigma, \tau, j \vDash \policy$ then $\rp{j}{\true} \in \histst{i}$ (Promptness)
.
\end{theorem}
\textit{Proof.} By  induction on the  formula $\policy$ and time step $i$. See \cite{Kane2015} \\

We now discuss the runtime complexity of \monitor while checking satisfaction of a property \policy. At any most recent position of the trace, let us assume we have maximum $L$ positions in the current trace for which \policy has not yet been reduced to a boolean value.
Note that if the value of $\wdelay(\policy)$ is large $L$ can be as large as the current trace length we have seen so far.
Addtionally, for each temporal subformula $\phi_1\until\phi_2$ of \policy,
we build history structures that
keep track of segments of the trace positions for which $\phi_1$ is true. Let  assume
we have $S$ such segments that are relevant (relevant means they can possibly
influence the satisfaction of the remaining unreduced formulas).
Hence, the complexity of \monitor is
$\mathcal{O}(LM|\policy|)$.

%Theorem \ref{thm:eagerness} states that \monitor is eager.
%
%\begin{theorem}[Eagerness of \monitor]
%\label{thm:eagerness}
%For all $i \in \mathbb{N}$, all formula $\varphi$, all time stamp sequences $\tau$ and all traces $\sigma$ it is the case that (1) if $\sigma_0,\ldots,\sigma_i$ is an informative prefix then if $\sigma,\tau, i \vDash \policy$ then $\rpt{i}{\true}{\policy} \in \histst{i}$ and if $\sigma, \tau, i \nvDash \policy$ then $\rpt{i}{\false}{\policy} \in \histst{i}$.
%\end{theorem}
%\textit{Proof.} By mutual induction on the policy formula $\policy$ and time step $i$. See \cite{TechPaper}
%%%%%%%%%%%%%%%%%%%%%%%%%%% REWRITE LINE %%%%%%%%%%%%%%%%%%%%%
%%%%%%%%%%%%%%%%%%%%%%%%%%% REWRITE LINE %%%%%%%%%%%%%%%%%%%%%
